% This snippet resets the row counter in tabularx so it can work correctly with rowcolors.
\newcounter{tblerows}
\expandafter\let\csname c@tblerows\endcsname\rownum

\begin{table*}[ht]
    \caption{List of surveyed papers in which a URL related to a tool has been found.}
    \label{tab:table:tools:urls}
    \centering
    \small
    \rowcolors{2}{white}{gray!25}
    \begin{tabularx}{\textwidth}{lXX}
	    \toprule
	    Reference & URL & Observations \\
	    \midrule
	    \cite{SIR}                                & \url{http://sir.unl.edu}                                          & This is a software repository. It is not a tool for amplification but it is a resource that could be used for amplification.\\
	    \cite{Baudry:2006:ITS:1134285.1134299}    & \url{http://www.irisa.fr/triskell/results/Diagnosis/index.htm}    & The URL points only to results. \\
	    \cite{bohme2014corebench}                 & \url{http://www.comp.nus.edu.sg/~release/corebench/}              & The website also contains empirical results.\\
	    \cite{Carzaniga:2014:COI:2568225.2568287} & \url{http://www.inf.usi.ch/phd/goffi/crosscheckingoracles/}       & \\
	    \cite{Dallmeier2010}                      & \url{https://www.st.cs.uni-saarland.de/models/tautoko/index.html} & \\
	    \cite{reassert2009}                       & \url{http://mir.cs.illinois.edu/reassert/}                        & \\
	    \cite{fang2015perfblower}                 & \url{https://bitbucket.org/fanglu/perfblower-public}              & There is no explicit url in the paper but a sentence saying that the tool is available in Bitbucket. With this information it was easy to find the URL. \\
	    \cite{fraser2011evosuite}                 & \url{http://www.evosuite.org/}                                    & Additional materials included. \\
	    \cite{marri2010retrofitting}              & \url{https://sites.google.com/site/asergrp/projects/putstudy}     & The website also contains empirical results.\\
	    \cite{milani2014}                         & \url{https://github.com/saltlab/Testilizer}                       & \\
	    \cite{Pacheco2005}                        & \url{http://groups.csail.mit.edu/pag/eclat/}                      & The website provides basic usage example.\\
	    \cite{palikareva2016shadow}               & \url{https://srg.doc.ic.ac.uk/projects/shadow/}                   & The website also contains empirical results.\\
	    \cite{pezze2013}                          & \url{http://puremvc.org/}                                         & The paper has been turned into a company. The provided url is the url of this company.\\
	    \cite{robetaler2012isolating}             & \url{https://www.st.cs.uni-saarland.de/bugex/}                    & The url lives, but there is no way to download and try the tools. \\
	    \cite{xuan:hal-01309004}                  & \url{https://github.com/Spirals-Team/banana-refactoring}          & \\
	    \cite{xuanTSE2016Nopol}                   & \url{https://github.com/SpoonLabs/nopol}                          & Still active. \\
	    \cite{Zhang2016Isomorphic}                & \url{https://github.com/sei-pku/Ison}                             & \\
	    \bottomrule
	\end{tabularx}
\end{table*}