\subsection{Example}
\label{subsec:sota:category-4:example}

\begin{lstlisting}[caption={Example of a toy class},label=lst:example:amplification:original,float,language=java,numbers=left]
public class Stack {
	private Comparable[] elems;
	public Stack() { ... }
	public void push(Comparable i) { ... }
	public void pop() { ... }
	public boolean isFull() { ... }
	public boolean isEmpty() { ... }
}
\end{lstlisting}

\begin{lstlisting}[caption={Initial test suite for the toy class},label=lst:example:test:initial,float,language=java,numbers=left]
public class StackTest {
	@Test
	public void test1() {  
		Stack s1 = new Stack();
		s1.push('a');
		s1.pop();
	}
}
\end{lstlisting}

\begin{lstlisting}[caption={Augmented test suite for the toy class},label=lst:example:test:aug,float,language=java,numbers=left]
public  class  StackTest {
	@Test
	public  void  testAug1 () {
		Stack s1 = new  Stack();
		assertTrue(s1.isEmpty());
		assertFalse(s1.isFull());
		s1.push('a');
		assertFalse(s1.isEmpty());
		assertFalse(s1.isFull());
		s1.pop();
	}
}
\end{lstlisting}

An example is given to illustrate the work of this category. 
Consider a simple Java class named \emph{Stack} in \autoref{lst:example:amplification:original}. 
The example is a simplified Java implementation of a stack that stores unique elements. 
In the implementation, the array \emph{elems} contains the elements of the stack, and the \emph{push} and \emph{pop} functions represent the two standard push and pop stack operations. 
The functions \emph{isFull} and \emph{isEmpty} check whether the stack is full and empty respectively.

Given the Java class, existing automatic test-generation tools can generate a test suite for it.
For instance, \autoref{lst:example:test:initial} exemplifies a possible test generated by automatic test-generation tools. 
Note however there are no assertions generated in the test suite. 
To detect problems during test execution, it typically relies on observing whether uncaught exceptions are thrown or whether the execution violates some predefined contracts. 

A test amplification technique $AMP_{mod}$ may be able to generate the amplified test suite as shown in \autoref{lst:example:test:aug}.
Compared with the original test suite, the augmented test suite has comprehensive assertions. 
These assertions reflect the behavior of the current program version under test and can be used to detect regression faults introduced in future program versions.
