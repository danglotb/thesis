\subsection{Example}
\label{subsec:sota:category-1:example}

In this section I present an example of $AMP_{add}$ to illustrate this category of work.
Let us consider the single Java method, presented in \autoref{lst:example}.

\begin{lstlisting}[caption={Example of a toy method},label=lst:example,float,language=java,numbers=left]
class Computer {
	public void compute(int integer) {
		if (integer > 2) {
			return integer + 2;
		} else {
			return integer + 1;
		}
	}
}
\end{lstlisting}

This method contains an if statement. 
The conditional expression tests the value passed through the parameter. 
If the value is greater than 2, then the method returns the value plus 2, otherwise it returns the value plus 1.
Applying $AMP_{add}$ requires to have existing tests. 
Consider the test method in \autoref{lst:example_test_method}.
This test method ensures the behavior of the program when the parameter is lower than 2, \ie when the else branch of the if statement is executed.

\begin{lstlisting}[caption={Example of toy test method},label=lst:example_test_method,float,language=java,numbers=left] 
@Test
public void test_compute() {
	Computer computer = new Computer();
	int actualValue = computer.compute(1);
	assertEquals(2, actualValue);
}
\end{lstlisting}

According to this test, one can say that this program is ``poorly'' tested, since only one of the two branches is covered.
One potential goal of an $AMP_{add}$ technique is to increase this branch coverage. 

\begin{lstlisting}[caption={Example of amplified toy test method},label=lst:example_test_method_amplified,float,language=java,numbers=left] 
@Test
public void amplified_test_compute() {
	Computer computer = new Computer();
	int actualValue = computer.compute(3);
	assertEquals(5, actualValue);
}
\end{lstlisting}

Now, an $AMP_{add}$ technique may be able to generate the amplified test method shown in \autoref{lst:example_test_method_amplified}.
The test \autoref{lst:example_test_method_amplified} is easily derivable from the existing test \autoref{lst:example_test_method} because only one literal and the assertion differ.
This new test method executes the \textit{then} branch of the if statement (see \autoref{lst:example} line 2 and 3) that was not executed before. 
That is to say, applying $AMP_{add}$ improves the test suite, by increasing the branch coverage of the program.