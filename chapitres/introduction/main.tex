\chapter{Introduction}
\label{chapt:introduction}

\minitoc

% software is everywhere
Nowadays software are omnipresent in people's life: banking, e-commerce, communication, etc. 
And the more and more, critical aspects such as aeronautics, voting system or even health care relies on software.

% errors can be dramatic
In one of his famous lectures~\cite{DijkstraLecture1989}, Dijkstra has stated, that in software, the following: 
\begin{center}
	\emph{``the smallest possible perturbations - \ie changes of a single bit - can have the most drastic consequences.''.}
\end{center}

Dijkstra highlights the fact that a small fault in software might affect lives.
For example, a flight crash happened in 1993 due to an error in the flight-control software of the the Swedish JAS 39 Gripen fighter aircraft.

% preventing bugs = testing tons of changes by day - CI
To avoid such situations, software editors adopted testing philosophies: the testers writes code that verify that the program is doing what the developer expects.
Over the last decade, strong unit testing has become an essential component of any serious software project, whether in industry or academia.
The agile development movement has contributed to this cultural change with the global dissemination of test-driven development techniques \cite{beck2003test}.
More recently, the DevOps movement has further strengthened the testing practice with an emphasis on continuous and automated testing \cite{Roche2013Devops}.

However, testing is tedious and costly for industries: there is no direct return to invest.
Thus, developers under pressure or by lack of discipline or time might skip the tests.

To overcome this problem, research investigates the automation of creating strong tests.
Automatic generation of tests has been well studied in the last past years~\cite{ESECFSE11, PachecoE2005}.
The dream was that a command-line would give you a complete test suite, that verifies the whole program.
For free, or almost, all the program would be well-tested.
However, studies show that developers are not using automatic test generation~\cite{TOSEM_userstudy}.
The authors investigate the truthfulness of the following hypothesis:

\emph{generating high coverage test data, we aid testers in constructing test suites capable of detecting faults.}

However, their studies showed that achieving high coverage does not necessarily improve the ability to test software.
The difficulties to understand, integrate and maintain automated test suite prevent the adoption by developers.
Also, most of the tools relies on weak or partial oracles, \eg absence of runtime errors, making them useless against bugs that do not set the program into a wrong state.

In this thesis, I aim at addressing this issue.
More precisely, the ultimate goal of this thesis is to provide an usable tool that assist developers to maintain their suite, in the context of DevOps and Continuous Integration.
To do so, I use test suite amplification, which is an emerging field, to create specific test methods according to an engineering goal.

The remainding of this section is as follow:

In \autoref{sec:intro:stamp} present the context of the thesis, the H2020 European project \emph{STAMP};

Then, I expose the roadmap of this thesis and its global vision in \autoref{sec:intro:roadmap};

Eventually, I list my publications in \autoref{sec:intro:publications} and the resulting software of this thesis in \autoref{sec:intro:software}.

\section{STAMP-project}
\label{sec:intro:stamp}

\section{Dissertation Roadmap}
\label{sec:intro:roadmap}

\section{Publications}
\label{sec:intro:publications}

\section{Software and Impact On The Community}
\label{sec:intro:software}