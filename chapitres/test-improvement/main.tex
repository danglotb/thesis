\chapter{Test Amplification For Artificial Behavioral Changes Detection Improvement}
\label{chap:test-improvement}

\minitoc

\graphicspath{{.}{chapitres/test-improvement/}}

This thesis aims at supporting developers in their developing tasks.
One of them is to evolve the test suite, \ie modify or add test methods, to strengthen their confidence in the program's correctness.
In \autoref{chap:dspot}, I introduced \dspot which is a test suite amplification tool.
In this chapter, I evaluate the effectiveness of \dspot to improve the quality of a test suite and the acceptability of the resulting amplified test methods.

This evaluation is based on the mutation score as test-criterion.
I confronted \dspot's output to real projects from \gh.
To do so, I proposed to developers to integrate directly the amplified test methods into their test suite.
Developers showed their interest in amplified test methods by permanently accepting some \dspot's amplified test methods into their test suite.
I also performed an evaluation on 40 test classes from 10 projects from \gh and showed that \dspot improves 26 of them.

To sum up, the contributions of this chapter are:
\begin{itemize}
	\item the design and execution of an experiment to assess the relevance of \dspot, based on feedback from the developers of mature and active projects;
	\item the design and execution of a large scale quantitative study of the improvement of 40 real-world test classes taken from 10 mature open-source Java projects.	
	\item fully open-science data: the experimental data are made publicly available for future research\footnote{\url{https://github.com/STAMP-project/dspot-experiments/}}
\end{itemize}
Note that this chapter has been published~\cite{Danglot2019} in the Springer journal \emph{Empirical Software Engineering} and the remainder is as follows:
\autoref{sec:test-improvement:mutation-score} introduces \ms as a test-criterion.
\autoref{sec:test-improvement:experiment-protocol} presents the experimental protocol of our study.
\autoref{sec:test-improvement:experiment-results} analyses our empirical results..
\autoref{sec:test-improvement:threats} discusses the threats to validity.
and \autoref{sec:test-improvement:conclusion} concludes this chapter.

\section{Mutation score as test-criterion}
\label{sec:test-improvement:mutation-score}

Mutation score measures the test suite's ability to detect artificial behavioral changes.
Briefly, it is measured as follow:

1) it injects a fault, or an artificial behavioral change, in the source code, \eg changes a $\ge$ into a $>$.
This modified program is called ``mutants''.
It generates different mutants with different artificial behavioral change;

2) it executes the test suite on the mutant;

3) it collects the result of the test suite execution.
If at least on test method fails, it means that the test suite is able to detect the fault.
It is said that the test suite kills the mutant;
If no test methods failed, it means that the test suite is not able to detect the fault. 
It is said that the mutant remains alive.

4) to compute the \ms, one must compute the percentage of mutants killed over the mutants generated.
The more mutants the test suite kills, the better is considered the test suite.

Mutation score aims at emulating faults that a developer could integrate in his code.
If the test suite has a high \ms, the probability that it detects such fault increase.

\dspot uses \pitest\cite{coles_pit_2016} \footnote{latest version released at the time of the experimentation: 1.2.0.\url{https://github.com/hcoles/pitest/releases/tag/1.2.0}} because: 

1) it targets Java programs;

2) it is mature and well-regarded;

3) it has an active community.

The most important feature of \pitest is that if the application code remains unchanged, the generated mutants are always the same.
This property is very interesting for test amplification.
Since \dspot only modifies test code, this feature allows us to compare the \ms of the original test method against the \ms of the amplified version and even compare the absolute number of mutants killed by two test method variants. 
\dspot exploits this feature to use \ms as a reliable test-criterion:
since \dspot never modifies the application code, the set of mutants is the same between runs and thus allow \dspot to a concrete and stable baseline for the baseline.
\dspot can compare mutants killed before and mutants killed after the amplification in order to select amplified test methods that kill mutants that were not killed by the original test suite.

By default, \dspot uses all the mutation operators available in \pitest: 
conditionals boundary mutator;
increments mutator;
invert negatives mutator;
math mutator;
negate conditionals mutator;
return values mutator;
void method calls mutator.
For more information, see the dedicated section of \pitest's website: \url{http://pitest.org/quickstart/mutators/}.

In this experimentation, \ms has been choose over coverage because \ms is consider stronger than coverage.
The purpose of test suites is to check the program's behavior.
In one hand, coverage is only based on the execution of the program and do not require any oracles.
Coverage does not measure the proportion of the behavior tested but only the proportion of code executed.
In the other hand, \ms requires oracles and thus to have a high \ms, the test suite must contains oracles.

\section{Experimental  Protocol}
\label{sec:test-improvement:experiment-protocol}

Recalling that \dspot is a automatic test improvement process.
Such processes have been evaluated with respect to evolutionary test inputs \cite{tonella} and new assertions \cite{Xie2006}.
However:

1) the two topics have never been studied in conjunction

2) they have never been studied on large modern Java programs

3) most importantly, the quality of improved tests has never been assessed by developers.

I set up a novel experimental protocol that addresses those three points.
First, the experiment is based on \dspot, which combines test input exploration and assertion generation.
Second, the experiment is made on 10 active \gh projects.
Third, I have proposed improved tests to developers under the form of pull-requests.

The evaluation aims at answering the following research questions:

\newcommand\rqpullrequest{RQ1\xspace}
\newcommand\rqcandidates{RQ2\xspace}
\newcommand\rqeffectiveness{RQ3\xspace}
\newcommand\rqAmplVersusIAmpl{RQ4\xspace}

\subsection{Research Questions}
\label{subsec:test-improvement:experiment-protocol:research questions}

\noindent\textbf{\rqpullrequest}: Are the improved test cases produced by \dspot relevant for developers? Are the developers ready to permanently accept the improved test cases into the test repository?\\
\textbf{\rqcandidates}: To what extent are improved test methods considered as focused?\\
\textbf{\rqeffectiveness}: To what extent do the improved test classes increase the \ms of the original,  manually-written, test classes?\\
\textbf{\rqAmplVersusIAmpl}: What is the relative contribution of \Iampl{} and \Aampl{} to the effectiveness of automatic test improvement?\\

\subsection{Dataset}
\label{subsec:test-improvement:experiment-protocol:dataset}

\dspot has been evaluated by amplifying test classes of large-scale, notable, open-source projects. 
The dataset includes projects that fulfil the following criteria:

1) the project must be written in Java; 

2) the project must have a test suite based on \junit;

3) the project must be compiled and tested with Maven;

4) the project must have an active community as defined by the presence of pull requests on \gh, see \autoref{subsec:test-improvement:experiment-results:rq1}. 

	\begin{table}[ht]
		
		\scriptsize
		\begin{tabular}{l|l|r|r|l}
			\label{tab:dataset}
			project & 
			description &
			\# LOC & \# PR &considered test classes\\
			\hline
			\rowcolor[HTML]{EEEEEE}
			javapoet & Java source file generator & 
			3150 & 93 &
			\begin{tabular}{@{}l@{}}
				TypeNameTest$^h$~~NameAllocatorTest$^h$\\FieldSpecTest$^l$~~ParameterSpecTest$^l$
			\end{tabular}\\
			mybatis-3 & Object-relational mapping framework &
			20683 & 288 &
			\begin{tabular}{@{}l@{}}
				MetaClassTest$^h$~~ParameterExpressionTest$^h$\\WrongNamespacesTest$^l$~~WrongMapperTest$^l$
			\end{tabular}\\
			\rowcolor[HTML]{EEEEEE}
			traccar & Server for GPS tracking devices &
			32648 & 373 &
			\begin{tabular}{@{}l@{}}
				GeolocationProviderTest$^h$~~MiscFormatterTest$^h$\\ObdDecoderTest$^l$~~At2000ProtocolDecoderTest$^l$
			\end{tabular}\\
			stream-lib & Library for summarizing data in streams &
			4767 & 21 &
			\begin{tabular}{@{}l@{}}
				TestLookup3Hash$^h$~~TestDoublyLinkedList$^h$\\TestICardinality$^l$~~TestMurmurHash$^l$
			\end{tabular}\\
			\rowcolor[HTML]{EEEEEE}
			mustache.java & Web application templating system &
			3166 & 11 &
			\begin{tabular}{@{}l@{}}
				ArraysIndexesTest$^h$~~ClasspathResolverTest$^h$\\ConcurrencyTest$^l$~~AbstractClassTest$^l$
			\end{tabular}\\
			twilio-java & Library for communicating with Twilio REST API &
			54423 & 87 &
			\begin{tabular}{@{}l@{}}
				RequestTest$^h$~~PrefixedCollapsibleMapTest$^h$\\AllTimeTest$^l$~~DailyTest$^l$
			\end{tabular}\\
			\rowcolor[HTML]{EEEEEE}
			jsoup & HTML parser &
			10925 & 72 &
			\begin{tabular}{@{}l@{}}
				TokenQueueTest$^h$~~CharacterReaderTest$^h$\\AttributeTest$^l$~~AttributesTest$^h$
			\end{tabular}\\
			protostuff& Data serialization library &
			4700 & 35 &
			\begin{tabular}{@{}l@{}}
				TailDelimiterTest$^h$~~LinkBufferTest$^h$\\CodedDataInputTest$^l$~~CodedInputTest$^h$
			\end{tabular}\\
			\rowcolor[HTML]{EEEEEE}
			logback & Logging framework &
			15490 & 104 &
			\begin{tabular}{@{}l@{}}
				FileNamePatternTest$^h$~~SyslogAppenderBaseTest$^h$\\FileAppenderResilience\_AS\_ROOT\_Test$^l$~~Basic$^l$
			\end{tabular}\\
			retrofit & HTTP client for Android. & 
			2743 & 249 &
			\begin{tabular}{@{}l@{}}
				RequestBuilderAndroidTest$^h$~~CallAdapterTest$^h$\\ExecutorCallAdapterFactoryTest$^h$~~CallTest$^h$
			\end{tabular}\\
		\end{tabular}
		\caption{Dataset of 10 active \gh projects considered on our relevance study (RQ1) and quantitative experiments (RQ2, RQ3).}
	\end{table}

Those criteria have been implemented as a query on top of TravisTorrent \cite{msr17challenge}. 
10 projects has been selected from the result of the query which composed the dataset presented in \autoref{tab:dataset}.
This table gives the project name, a short description, the number of pull-requests on \gh (\#PR), and the considered test classes.
For instance, \emph{javapoet} is a strongly-tested and active project, which implements a Java file generator, it has had 93 pull-requests in 2016.


\subsection{Test Case Selection Process}
\label{subsec:test-improvement:experiment-protocol:test-preparation}

For each project, 4 test classes have been select to be amplified. 
Those test classes are chosen as follows.

% Unit Tests
First, the test class must be a unit-test classes only, because \dspot focuses on unit test amplification. 
I use the following heuristic to discriminate unit test cases from others: 
test classes kept are test classes which executes less than an arbitrary threshold of S statements, \ie if it covers a small portion of the code.
In this experiment, $S=1500$.

Among the unit-tests, 4 classes has been selected as follows.
Since I want to analyze the performance of \dspot when it is provided with both good and bad tests, selected test classes has been split into two groups: 
one group with strong tests, one other group with low quality tests.
\ms has been used to distinguish between good and bad test classes.
Accordingly, the selection process has five steps:

1) Compute the original \ms of each class with \pitest (see \autoref{sec:test-improvement:mutation-score};

2) Discard test classes that have 100\% \ms, because they can already be considered as perfect tests 
(this is the case for eleven classes, showing that the considered projects in the dataset are really well-tested projects);

3) Sort the classes by \ms ( see \autoref{subsec:test-improvement:experiment-protocol:metrics}), in ascending order;

4) Split the set of test classes into two groups: high \ms( $> 50\%$) and low \ms  ($< 50\%$);

5) Randomly select 2 test classes in each group.

This selection results with 40 test classes: 24 in high mutation group score and 16 in low \ms group.
The imbalance is due to the fact that there are three projects really well tested for which there are none or a single test class with a low \ms (projects protostuff, jsoup, retrofit).
Consequently, those three projects are represented with 3 or 4 well-tested classes (and 1 or 0 poorly-tested class). 
In \autoref{tab:dataset}, the last column contains the name of the selected test classes. 
Each test class name is indexed by a ``h'' or a ``l'' which means respectively that the class have a high \ms or a low \ms.

\subsection{Metrics}
\label{subsec:test-improvement:experiment-protocol:metrics}

\textbf{Number of Killed Mutants} ($\#Killed.Mutants$): is the absolute number of mutants killed by a test class. 
It used to compare the fault detection power of an original test class and the one of its amplified version.

\textbf{Mutation Score}: is the percentage of killed mutants over the number of executed mutants.
 Mathematically, it is computed as follow: $$\frac{\#Killed.Mutants}{\#Exec.Mutants} \time 100$$.

\textbf{Increase Killed}: is the relative increase of the number of killed mutants by an original test class $T$ and the number of killed mutants by its amplified version $T_a$.
It is computed as follows:
$$\frac{\#Killed.Mutants_{T_a} - \#Killed.Mutants_T}{\#Killed.Mutants_T}$$
The goal of \dspot is to improve tests such that the number of killed mutants increases.

\subsection{Methodology}
\label{subsec:test-improvement:experiment-protocol:methodology}

This experimental protocol has been designed to study to what extent \dspot and its result are valuable for the developer.

\begin{itemize}
	\item \textbf{\rqpullrequest}
	% process
	To answer to \rqpullrequest, pull-requests have been created on notable open-source projects.
	\dspot amplifies 19 test classes of selected projects and I propose amplified test methods to the main developers of each project under consideration in the form of pull requests (PR) on \gh.
	A PR is composed of a title, a short text that describes the purpose of changes and a set of code change (aka a patch).
	The main developers review, discuss and decide to merge or not each pull request.
	I base the answer on the subjective and expert assessment from projects' developers.
	If a developer merges an improvement synthesized by \dspot, it validates the relevance of \dspot.
	The more developers accept and merge test improvements produced by \dspot into their test suite, the more the amplification is considered successful.
	
	\item \textbf{\rqcandidates}
	To answer \rqcandidates, the number of suggested improvements is computed, to verify that the developer is not overwhelmed with suggestions.
	The number of focused amplified test methods is computed following the technique described in \autoref{subsubsec:test-improvement:experiment-results:rq1:selection}, for each project in the benchmark.
	I present and discuss the proportion of focused tests out of all proposed amplified tests.
	
	\item \textbf{\rqeffectiveness}
	To answer \rqeffectiveness, I see whether the value that is taken as proxy to the developer value -- the \ms -- is appropriately improved.
	For 40 real-world classes, first \pitest (see \autoref{sec:test-improvement:mutation-score}) is ran the mutation testing tool on the test class. 
	This gives the \ams for this original class. 
	Then, the test class under consideration is amplified and the new \ams after amplification is computed. 
	Finally, the result are compared and analyzed.
	
	\item \textbf{\rqAmplVersusIAmpl}
	To answer \rqAmplVersusIAmpl, the number of \Aampl and \Iampl amplifications are computed. 
	The former means that the suggested improvement is very short hence easy to be accepted by the developer while the latter means that more time would be required to understand the improvement.
	First, I collect three series of metrics: 
	
	1) I compute \ams for the original test class;
	
	2) I improve the test class under consideration using only \Aampl{} and compute the new \ams after amplification; 
	
	3) I improve the test class under consideration using \Iampl{} as well as \Aampl{} (the standard complete \dspot workflow) and compute the \ams after amplification. 
	
	Then, I compare the increase of \ms obtained by using \Aampl{} only and \Aampl{} + \Iampl{}.\footnote{Note that the relative contribution of \Iampl{} cannot be evaluated alone, because as soon as \dspot modifies the inputs in a test case, it is also necessary to change and improve the oracle (which is the role of \Aampl{}).}
\end{itemize}

Research questions 3 and 4 focus on the \ms to assess the value of amplified test methods.
This experimental design choice is guided by the approach to select ``focused'' test methods, which are likely to be selected by the developers (described in \autoref{subsubsec:test-improvement:experiment-results:rq1:selection}). 
Recall that the number of killed mutants by the amplified test is the key focus indicator. 
Hence, the more \dspot is able to improve the \ms, the more likely there are good candidates for the developers.

\section{Experimental Results}
\label{sec:test-improvement:experiment-results}

\subsection{Answer to \rqpullrequest}
\label{subsec:test-improvement:experiment-results:rq1}

\textbf{\rqpullrequest: Would developers be ready to permanently accept automatically improved test cases into the test repository?}

\subsubsection{Process}
\label{subsubsec:test-improvement:experiment-results:rq1:process}

In this research question, the goal is to propose a new test to the lead developers of the open-source projects under consideration. 
The improved test is proposed through a ``pull-request'', which is a way to reach developers with patches on collaborative development platforms such as \gh.

In practice, short pull requests (\ie with small test modifications) with clear purpose, \ie what for it has been opened, have much more chance of being reviewed, discussed and eventually merged. 
So the goal is to provide improved tests which are easy to review.
As shown in \autoref{subsec:dspot:algorithm:input-space-exploration}, \dspot generates several amplified test cases, and all of them cannot be proposed to the developers.
To select the new test case to be proposed as a pull request, I look for an amplified test that kills mutants located in the same method.
From the developer's viewpoint, it means that the intention of the test is clear: it specifies the behavior provided by a given method or block.

\subsubsection{Selection Of Amplified Method For Pull Requests}
\label{subsubsec:test-improvement:experiment-results:rq1:selection}

\dspot sometimes produces many tests, from one initial test.
Due to limited time, the developer needs to focus on the most interesting ones.
To select the test methods that are the most likely to be merged in the code base, the following heuristic is implemented:
First, the amplified test methods are sorted according to the ratio of newly killed mutants and the total number of test modifications.
Then, in case of equality, the methods are further sorted according to the maximum numbers of mutants killed in the same method.

The first criterion means that short modifications have more valuable than large ones.
The second criterion means that the amplified test method is focused and tries to specify one specific method inside the code.

If an amplified test method is merged in the code base, the corresponding method is considered as specified. 
In that case, other amplified test methods that specify the same method are no longer taken into account.

Finally, in this ordered list, the developer is recommended the amplified tests that are focused, where focus is defined as where at least 50\% of the newly killed mutants are located in a single method. 
The goal is to select amplified tests which intent can be easily grasped by the developer: the new test specifies the method.

For each selected method, I compute and minimize the diff between the original method and the amplified one and then the diff as a pull request is submitted.
A second point in the preparation of the pull request relates to the length of the amplified test: 
once a test method has been selected as a candidate pull request, the diff is made as concise as possible for the review to be fast and easy.

\subsubsection{Overview}

\begin{table}[!h]
	\label{tab:res-pr}
	\centering
	\rowcolors{2}{white}{gray!25}
	\caption{Overall result of the opened pull request built from result of \dspot.}
	\begin{tabular}{lcccc}
		\toprule
		Project & \# opened & \# merged & \# closed & \begin{tabular}{cc} \# under \\ discussion \end{tabular}\\
		\midrule
		javapoet & 4 & 4 & 0 & 0\\
		mybatis-3 & 2 & 2 & 0 & 0\\
		traccar & 2 & 1 & 0 & 1\\
		stream-lib & 1 & 1 & 0 & 0\\
		mustache & 2 & 2 & 0 & 0\\
		twilio & 2 & 1 & 0 & 1\\
		jsoup & 2 & 1 & 1 & 0\\
		prostostuff & 2 & 2 & 0 & 0\\
		logback & 2 & 0 & 0 & 2\\
		retrofit & 0 & 0 & 0 & 0\\
		\midrule
		total & 19 & 14 & 1 & 4\\
		\bottomrule
	\end{tabular}
\end{table}

In total, 19 pull requests has been created, as shown in \autoref{tab:res-pr}. 
In this table, the first column is the name of the project, the second is number of opened pull requests, \ie the number of amplified test methods proposed to developers. 
The third column is the number of amplified test methods accepted by the developers and permanently integrated in their test suite. 
The fourth column is the number of amplified test methods rejected by the developers. 
The fifth column is the number of pull requests that are still being discussed, \ie nor merged nor closed. 
Note that these numbers might change over time if pull-requests are merged or closed.

\begin{table}[!h]
    \rowcolors{2}{white}{gray!25}
    \centering
	\label{tab:list-urls-prs}
	\caption{List of URLs to the pull-requests created in this experiment.}
	\begin{tabular}{ll}
		\toprule
		Project & Pull request URLs\\
		\midrule
		javapoet & 
		\begin{tabular}{l}
			\url{https://github.com/square/javapoet/pull/669}\\
			\url{https://github.com/square/javapoet/pull/668}\\
			\url{https://github.com/square/javapoet/pull/667}\\ 
			\url{https://github.com/square/javapoet/pull/544}
		\end{tabular} \\
		mybatis-3 & 
		\begin{tabular}{l}
			\url{https://github.com/mybatis/mybatis-3/pull/1331}\\
			\url{https://github.com/mybatis/mybatis-3/pull/912}
		\end{tabular} \\
		traccar &
		\begin{tabular}{l}
			\url{https://github.com/traccar/traccar/pull/2897}\\
			\url{https://github.com/traccar/traccar/pull/4012}
		\end{tabular} \\
		stream-lib &
		\begin{tabular}{l}
			\url{https://github.com/addthis/stream-lib/pull/128}\\
		\end{tabular} \\
		mustache &
		\begin{tabular}{l}
			\url{https://github.com/spullara/mustache.java/pull/210}\\
			\url{https://github.com/spullara/mustache.java/pull/186}
		\end{tabular} \\
		twilio &
		\begin{tabular}{l}
			\url{https://github.com/twilio/twilio-java/pull/437}\\
			\url{https://github.com/twilio/twilio-java/pull/334}
		\end{tabular} \\
		jsoup &
		\begin{tabular}{l}
			\url{https://github.com/jhy/jsoup/pull/1110}\\
			\url{https://github.com/jhy/jsoup/pull/840}
		\end{tabular} \\
		protostuff &
		\begin{tabular}{l}
			\url{https://github.com/protostuff/protostuff/pull/250}\\
			\url{https://github.com/protostuff/protostuff/pull/212}
		\end{tabular} \\
		logback &
		\begin{tabular}{l}
			\url{https://github.com/qos-ch/logback/pull/424}\\
			\url{https://github.com/qos-ch/logback/pull/365}
		\end{tabular} \\
		\bottomrule
	\end{tabular}
\end{table}

Overall 14 over 19 have been merged. 
Only 1 has been rejected by developers. 
There are 4 under discussion.
\autoref{tab:list-urls-prs} contains the URLs of pull requests proposed in this experimentation.

In the following, one pull-request per project is analyzed.

\subsubsection{javapoet}

\dspot has been applied to amplify \texttt{TypeNameTest}. 
\dspot synthesizes a single assertion that kills 3 more mutants, all of them at line 197 of the equals method. 
A manual analysis reveals that this new assertion specifies a contract for the method \texttt{equals()} of objects of type \texttt{TypeName}: 
the method must return false when the input is null. 
This contract was not tested.

Consequently, I have proposed to the Javapoet developers one liner pull request \footnote{\url{https://github.com/square/javapoet/pull/544}} showed in \autoref{lst:pr-javapoet}.
\begin{lstlisting}[language=diff,caption=Test-improvement proposed to Javapoet developers.,label=lst:pr-javapoet]
@@ -178,5 +179,6 @@ private void assertEqualsHashCodeAndToString(TypeName a, TypeName b) {
  assertEquals(a.toString(), b.toString());
  assertThat(a.equals(b)).isTrue();
  assertThat(a.hashCode()).isEqualTo(b.hashCode());
+ assertFalse(a.equals(null));
\end{lstlisting}

The title of the pull resuest is: ``\emph{Improve test on TypeName}'' with the following short text: ``\emph{Hello, I open this pull request to specify the line 197 in the equals() method of com.squareup.javapoet.TypeName. if (o == null) return false;}''
This test improvement synthesized by \dspot has been merged by of the lead developer of javapoet one hour after its proposal.

\subsubsection{mybatis-3}

In project mybatis-3, \dspot has been applied to amplify a test for \texttt{MetaClass}. 
\dspot synthesizes a single assertion that kills 8 more mutants.
All new mutants killed are located between lines 174 and 179, \ie the \texttt{then} branch of an \texttt{if-statement} in method \texttt{buildProperty(String property, StringBuilder sb)} of \texttt{MetaClass}.
This method builds a String that represents the  property given as input. 
The \texttt{then} branch is responsible to build the String in case the \texttt{property} has a child, \eg the input is ``richType.richProperty''. 
This behavior is not specified at all in the original test class.

I proposed to the developers the pull request, showed in \autoref{lst:pr-mybatis}~entitled ``\emph{Improve test on MetaClass}'' with the following short text: ``\emph{Hello, I open this pull request to specify the lines 174-179 in the buildProperty(String, StringBuilder) method of MetaClass.}'', \footnote{\url{https://github.com/mybatis/mybatis-3/pull/912/files}}

\begin{lstlisting}[language=diff,caption=Test-improvement proposed to MyBatis-3 developers.,label=lst:pr-mybatis]
@@ -65,6 +65,8 @@ public void shouldCheckGetterExistance() {
  assertTrue(meta.hasGetter("richType.richMap"));
  assertTrue(meta.hasGetter("richType.richList[0]"));
        
+ assertEquals(
+   richType.richProperty", 
+   meta.findProperty("richType.richProperty", false)
+ );
\end{lstlisting}

The developer accepted the test improvement and merged the pull request the same day without a single objection. 

\subsubsection{traccar}

\dspot has been applied to amplify \texttt{ObdDecoderTest}. 
It identifies a single assertion that kills 14 more mutants.
All newly killed mutants are located between lines 60 to 80, \ie in the method \texttt{decodesCodes()} of \texttt{ObdDecoder}, which is responsible to decode a \texttt{String}. 
In this case, the pull request consists of a new test method because the new assertions do not fit with the intent of existing tests. 
This new test method is proposed into \texttt{ObdDecoderTest}, which is the class under amplification. 
The PR was entitled ``\emph{Improve test cases on ObdDecoder}'' with the following description: ``\emph{Hello, I open this pull request to specify the method decodeCodes of the ObdDecoder}''. \footnote{\url{https://github.com/tananaev/traccar/pull/2897}}
The PR is shown in \autoref{lst:pr-traccar}.

\begin{lstlisting}[language=diff,caption=Test-improvement proposed to traccar developers.,label=lst:pr-traccar]
@@ -16,4 +16,10 @@ public void testDecode() {

  }

+  @Test
+  public void testDecodeCodes() throws Exception {
+    Assert.assertEquals("P0D14", ObdDecoder.decodeCodes("0D14").getValue());
+    Assert.assertEquals("dtcs", ObdDecoder.decodeCodes("0D14").getKey());
+  }
\end{lstlisting}

The developer of traccar thanked us for the proposed changes and merged it the same day.

\subsubsection{stream-lib}

\dspot has been applied to amplify \texttt{TestMurmurHash}. 
It identifies a new test input that kills 15 more mutants.
All newly killed mutants are located in method \texttt{hash64()} of \texttt{MurmurHash} from lines 158 to 216.
This method computes a hash for a given array of byte. 
The PR, shown in \autoref{lst:pr-stream-lib}, was entitled ``\emph{Test: Specify hash64}'' with the following description: ``\emph{The proposed change specifies what the good hash code must be. With the current test, any change in "hash" would still make the test pass, incl. the changes that would result in an inefficient hash.}''. \footnote{\url{https://github.com/addthis/stream-lib/pull/127/files}}

\begin{lstlisting}[language=diff,caption=Test-improvement proposed to stream-lib developers.,label=lst:pr-stream-lib]
@@ -44,7 +44,7 @@ public void testHash64ByteArrayOverload() {
  String input = "hashthis";
  byte[] inputBytes = input.getBytes();

- long hashOfString = MurmurHash.hash64(input);
+ long hashOfString = -8896273065425798843L;
  assertEquals("MurmurHash.hash64(byte[]) did not match MurmurHash.hash64(String)",
  hashOfString, MurmurHash.hash64(inputBytes));
\end{lstlisting}

Two days later, one developer mentioned the fact that the test is verifying the overload of the method and is not specifying the method hash itself. 
He closed the PR because it was not relevant to put changes there. 
He suggested to open an new pull request with a new test method instead of changing the existing test method. 
I proposed, 6 days later, a second pull request entitled ``\emph{add test for hash() and hash64() against hard coded values}'' with no description, since I estimated that the developer was aware of the test intention.\footnote{\url{https://github.com/addthis/stream-lib/pull/128/files}}.
This second pull request is shown in \autoref{lst:pr-stream-lib-2}.

\begin{lstlisting}[float,language=diff,caption=Test-improvement proposed to stream-lib developers with developers' suggestions.,label=lst:pr-stream-lib-2]
@@ -52,4 +52,22 @@ public void testHash64ByteArrayOverload() {
assertEquals("MurmurHash.hash64(Object) given a byte[] did not match MurmurHash.hash64(String)",
hashOfString, MurmurHash.hash64(bytesAsObject));
}

+   // test the returned valued of hash functions against
+   // the reference implementation: https://github.com/aappleby/smhasher.git

+   @Test
+   public void testHash64() throws Exception {
+     final long actualHash = MurmurHash.hash64("hashthis");
+     final long expectedHash = -8896273065425798843L;
+
+     assertEquals(
+       "MurmurHash.hash64(String) returns wrong hash value",
+       expectedHash,
+       actualHash
+     );
+   }

+   @Test
+   public void testHash() throws Exception {
+     final long actualHash = MurmurHash.hash("hashthis");
+     final long expectedHash = -1974946086L;
+
+     assertEquals(
+       "MurmurHash.hash64(String) returns wrong hash value",
+       expectedHash,
+       actualHash
+     );
+   }
\end{lstlisting}

The pull request has been merged by the same developer 20 days later.

\subsubsection{mustache.java}
\label{subsubsec:test-improvement:experiment-results:rq1:mustache}

\dspot has been applied to amplify \texttt{AbstractClassTest}. 
It identifies a try/catch/fail block that kills 2 more mutants.
This is an interesting new case, compared to the ones previously discussed, because it is about the specification of exceptions, \ie of behavior under erroneous inputs.
All newly killed mutants are located in method \texttt{compile()} on line 194.
The test specifies that if a variable is improperly closed, the program must throw a \texttt{MustacheException}. 
In the Mustache template language, an improperly closed variable occurs when an opening brace ``$\{$'' does not have its matching closing brace such as in the input of the proposed changes. 
I propose the pull request, shown \autoref{lst:pr-mustache.java}, to the developers, entitled ``\emph{Add Test: improperly closed variable}'' with the following description: ``\emph{Hello, I proposed this change to improve the test on MustacheParser. When a variable is improperly closed, a MustacheException is thrown.}''.\footnote{\url{https://github.com/spullara/mustache.java/pull/186/files}}

\begin{lstlisting}[language=diff,caption=Test-improvement proposed to mustache.java developers.,label=lst:pr-mustache.java]
@@ -63,4 +66,15 @@ public void testAbstractClassNoDots() throws IOException {
    mustache.execute(writer, scopes);
    writer.flush();
  }

+ @Test
+ public void testImproperlyClosedVariable() throws IOException {
+   try {
+     new DefaultMustacheFactory()
+       .compile(new StringReader("{{{#containers}} {{/containers}}"), "example");
+     fail("Should have throw MustacheException");
+   } catch (MustacheException actual) {
+     assertEquals(
+       "Improperly closed variable in example:1 @[example:1]",
+       actual.getMessage()
+     );
+   }
+ }
\end{lstlisting}

12 days later, a developer accepted the change, but noted that the test should be in another class.
He closed the pull request and added the changes himself into the desired class.\footnote{the diff is same:\url{https://github.com/spullara/mustache.java/commit/9efa19d595f893527ff218683e70db2ae4d8fb2d}}. 

\subsubsection{twilio-java}

\dspot has been applied to amplify \texttt{RequestTest}. 
It identifies two new assertions that kill 4 more mutants. 
All killed mutants are between lines 260 and 265 in the method \texttt{equals()} of \texttt{Request}. 
The change specifies that an object \texttt{Request} is not equal to null nor an object of different type, \ie \texttt{Object} here. 
The pull request was entitled ``\emph{add test equals() on request}'', accompanied with the short description ``\emph{Hi, I propose this change to specify the equals() method of com.twilio.http.Request, against object and null value}''. \footnote{\url{https://github.com/twilio/twilio-java/pull/334/files}}
\autoref{lst:pr-twilio-java} shows this pull request.

\begin{lstlisting}[float,language=diff,caption=Test-improvement proposed to twilio-java developers.,label=lst:pr-twilio-java]
@@ -166,5 +166,13 @@ public void testRequiresAuthentication() {
+   assertTrue(request.requiresAuthentication());
+ }

+ @Test
+ public void testEquals() {
+   Request request = new Request(HttpMethod.DELETE, "/uri");
+   request.setAuth("username", "password");
+   assertFalse(request.equals(new Object()));
+   assertFalse(request.equals(null));
+ }
\end{lstlisting}

A developer merged the change 4 days later.

\subsubsection{jsoup}

\dspot has been applied to amplify \texttt{AttributeTest}. 
It identifies one assertion that kills 13 more mutants.
All mutants are in the method \texttt{hashcode} of Attribute. 
The pull request, shown in \autoref{lst:pr-jsoup}, was entitled ``\emph{add test case for hashcode in attribute}'' with the following short description ``\emph{Hello, I propose this change to specify the hashCode of the object org.jsoup.nodes.Attribute.}''\footnote{\url{https://github.com/jhy/jsoup/pull/840}}:

\begin{lstlisting}[language=diff,caption=Test-improvement proposed to jsoup developers.,label=lst:pr-jsoup]
@@ -17,4 +17,11 @@
assertEquals(s + "=\"A" + s + "B\"", attr.html());
assertEquals(attr.html(), attr.toString());
+ }

+ @Test
+ public void testHashCode() {
+   String s = new String(Character.toChars(135361));
+   Attribute attr = new Attribute(s, (("A" + s) + "B"));
+   assertEquals(111849895, attr.hashCode());
+ }
\end{lstlisting}

One developer highlighted the point that the \texttt{hashCode} method is an implementation detail, and it is not a relevant element of the API. 
Consequently, he did not accept our test improvement.

At this point, I have made two pull requests targeting \texttt{hashCode} methods. 
One accepted and one rejected. 
\texttt{hashCode} methods could require a different testing approach to validate the number of potential collisions in a collection of objects rather than checking or comparing the values of a few objects created for one explicit test case.
The different responses obtained reflect the fact that developer teams and policies ultimately decide how to test the hash code protocol and the outcome could be different from different projects.

\subsubsection{protostuff}

\dspot has been applied to amplify \texttt{TailDelimiterTest}. 
It identifies a single assertion that kills 3 more mutants.
All new mutants killed are in the method \texttt{writeTo} of \texttt{ProtostuffIOUtil} on lines 285 and 286, which is responsible to write a buffer into a given scheme. 
I proposed a pull request entitled ``\emph{assert the returned value of writeList}'', with the following short description ``\emph{Hi, I propose the following changes to specify the line 285-286 of io.protostuff.ProtostuffIOUtil.}''\footnote{\url{https://github.com/protostuff/protostuff/pull/212/files}}, shown in \autoref{lst:pr-protostuff}.

\begin{lstlisting}[language=diff,caption=Test-improvement proposed to protostuff developers.,label=lst:pr-protostuff]
@@ -144,7 +144,8 @@ public void testEmptyList() throws Exception
    ArrayList<Foo> foos = new ArrayList<Foo>();

    ByteArrayOutputStream out = new ByteArrayOutputStream();
-   writeListTo(out, foos, SerializableObjects.foo.cachedSchema());
+   final int bytesWritten =
+     writeListTo(out, foos, SerializableObjects.foo.cachedSchema()
+   );
+   assertEquals(0, bytesWritten);
    byte[] data = out.toByteArray();
\end{lstlisting}

A developer accepted the proposed changes the same day.

\subsubsection{logback}

\dspot has been applied to amplify \texttt{FileNamePattern}. 
It identifies a single assertion that kills 5 more mutant. 
Newly killed mutants were located at lines 94, 96 and 97 of the \texttt{equals} method of the \texttt{FileNamePattern} class. 
The proposed pull request was entitle ``\emph{test: add test on equals of FileNamePattern against null value}'' with the following short description: ``\emph{Hello, I propose this change to specify the equals() method ofFileNamePattern against null value}''.\footnote{\url{https://github.com/qos-ch/logback/pull/365/files}}:

\begin{lstlisting}[language=diff,caption=Test-improvement proposed to logback developers.,label=lst:pr-logback]
@@ -189,4 +190,11 @@ public void settingTimeZoneOptionHasAnEffect() {
    FileNamePattern fnp = new FileNamePattern("%d{hh, " + tz.getID() + "}", context);
    assertEquals(tz, fnp.getPrimaryDateTokenConverter().getTimeZone());
  }

+ @Test
+ public void testNotEqualsNull() {
+   FileNamePattern pp = new FileNamePattern("t", context);
+   assertFalse(pp.equals(null));
++ }
\end{lstlisting}

Even if the test asserts the contract that the \texttt{FileNamePattern} is not equals to null, and kills 5 more mutants, the lead developer does not get the point to test this behavior. 
The pull request has not been accepted.

\subsubsection{retrofit}

I did not manage to create a pull request based on the amplification of the test suite of retrofit. 
According to the result, the newly killed mutants are spread over all the code, and thus the amplified methods did not identify a missing contract specification. 
This could be explained by two facts: 
1) the original test suite of retrofit is strong: there is no test class with low \ms and a lot of them are very high \ms, \ie 90\% and more;
2) the original test suite of retrofit uses complex test mechanism such as mock and fluent assertions of the form the \texttt{assertThat().isSomething()}. 
For the former point, it means that \dspot has been able to improve, even a bit, the \ms of a very strong test suite, but not in targeted way that makes sense in a pull request.
For the latter point, this puts in evidence the technical challenge of amplifying fluent assertions and mocking mechanisms.

\subsubsection{Contributions of \Aampl and \Iampl to the Pull-requests}

\begin{table}[]
    \rowcolors{2}{white}{gray!25}
	\caption{Contributions of \Aampl and \Iampl on the amplified test method used to create a pull request.}
	\label{tab:contrib-a-i-ampl}
	\centering
    \begin{adjustbox}{max width=\textwidth,center=\textwidth}
		\begin{tabular}{lcc}
			\toprule
			Project & \#\Aampl &  \#\Iampl \\
			\midrule
			javapoet & 2 & 2 \\
			mybatis-3 & 3 & 3 \\
			traccar & 10 & 7 \\
			stream-lib & 2 & 2 \\
			mustache & 4 & 3 \\
			twilio & 3 & 4 \\
			jsoup & 34 & 0 \\
			protostuff & 1 & 1 \\
			logback & 2 & 2 \\
			\bottomrule
		\end{tabular}
	\end{adjustbox}
\end{table}

\autoref{tab:contrib-a-i-ampl} summarizes the contribution of \Aampl and \Iampl, where a contribution means an source code modification added during the main amplification loop. 
In 8 cases over the 9 pull-requests, both \Aampl and \Iampl were necessary. 
Only the pull request on jsoup was found using only \Aampl. 
This means that for all the other pull-requests, the new inputs were required to be able: 
1) to kill new mutants and 
2) to obtain amplified test methods that have values for the developers.

Note that this does not contradict with the fact that the pull requests are one-liners.
Most one-liner pull requests contain both a new assertion and a new input. Consider the following Javapoet's one liner \texttt{assertFalse(x.equals(null))} (javapoet). 
In this example, although there is a single line starting with ``assert'', there is indeed a new input, the value ``null''.

\begin{mdframed}
	\textit{\rqpullrequest: Would developers be ready to permanently accept improved test cases into the test repository?}\\
	Answer: 19 test improvements have been proposed to developers of notable open-source projects. 
	13/19 have been considered valuable and have been merged into the main test suite. 
	The developers' feedback has confirmed the relevance, and also the challenges of automated test improvement.
\end{mdframed}

In the area of automatic test improvement, this experiment is the first to put real developers in the loop, by asking them about the quality of automatically improved test cases.
To the best of my knowledge, this is the first public report of automatically improved tests accepted by unbiased developers and merged in the master branch of open-source repositories.

\begin{table}
	\caption{The effectiveness of test amplification with \dspot on 40 test classes: 24 well-tested (upper part) and 16 average-tested (lower part) real test classes from notable open-source Java projects.}
	%A star after a test name means that the test is considered as subject for the relevance study of \rqpullrequest.}
	\label{tab:overall-results}
	\def\arraystretch{0.55}%  1 is the default, change whatever you need
	\setlength\tabcolsep{0.45pt} % default value: 6pt
	\small
	\begin{tabular}{|llrrrr|rrrr|rrr|r|}
		\rotverticalinv{ID}&
		\rotverticalinv{Class}&
		\rotverticalinv{\# Orig. test methods}&
		\rotverticalinv{Mutation Score}&
		\rotverticalinv{\# New test methods}&
		\rotverticalinv{\begin{tabular}{l}
				Candidates \\ for pull request
		\end{tabular}}&
		\rotverticalinv{\# Killed mutants orig.}&
		\rotverticalinv{\# Killed mutants ampl.}&
		\rotverticalinv{Increase killed}&
		&%arrow for killed
		\setlength{\tabcolsep}{0cm} 
		\rotverticalinv{\begin{tabular}{l}
				\# Killed mutants only\\ A-ampl
		\end{tabular}}&
		\setlength{\tabcolsep}{0cm} 
		\rotverticalinv{\begin{tabular}{l}
				Increase killed only\\ A-ampl
		\end{tabular}}&
		&%arrow for killed
		\rotverticalinv{Time (minutes)}\\
		\hline\\
		&\multicolumn{3}{l}{High \ms}\\
		\hline\\
		1&\scriptsize{TypeNameTest}&12&50\%&19&8&599&715&19\%&{\color{ForestGreen}$\nearrow$}&599&0.0\%&$\rightarrow$&11.11 \\
		\rowcolor[HTML]{EFEFEF}
		2&\scriptsize{NameAllocatorTest}&11&87\%&0&0&79&79&0.0\%&$\rightarrow$&79&0.0\%&$\rightarrow$&4.76 \\
		3&\scriptsize{MetaClassTest}&7&58\%&108&10&455&534&17\%&{\color{ForestGreen}$\nearrow$}&455&0.0\%&$\rightarrow$&235.71 \\
		\rowcolor[HTML]{EFEFEF}
		4&\scriptsize{ParameterExpressionTest}&14&91\%&2&2&162&164&1\%&{\color{ForestGreen}$\nearrow$}&162&0.0\%&$\rightarrow$&25.93 \\
		5&\scriptsize{ObdDecoderTest}&1&80\%&9&2&51&54&5\%&{\color{ForestGreen}$\nearrow$}&51&0.0\%&$\rightarrow$&2.20 \\
		\rowcolor[HTML]{EFEFEF}
		6&\scriptsize{MiscFormatterTest}&1&72\%&5&5&42&47&11\%&{\color{ForestGreen}$\nearrow$}&42&0.0\%&$\rightarrow$&1.21 \\
		7&\scriptsize{TestLookup3Hash}&2&95\%&0&0&464&464&0.0\%&$\rightarrow$&464&0.0\%&$\rightarrow$&6.76 \\
		\rowcolor[HTML]{EFEFEF}
		8&\scriptsize{TestDoublyLinkedList}&7&92\%&1&1&104&105&0.97\%&{\color{ForestGreen}$\nearrow$}&104&0.0\%&$\rightarrow$&3.03 \\
		9&\scriptsize{ArraysIndexesTest}&1&53\%&15&4&576&647&12\%&{\color{ForestGreen}$\nearrow$}&586&1\%&{\color{ForestGreen}$\nearrow$}&10.58 \\
		\rowcolor[HTML]{EFEFEF}
		10&\scriptsize{ClasspathResolverTest}&10&67\%&0&0&50&50&0.0\%&$\rightarrow$&50&0.0\%&$\rightarrow$&4.18 \\
		11&\scriptsize{RequestTest}&17&81\%&4&3&141&156&10\%&{\color{ForestGreen}$\nearrow$}&141&0.0\%&$\rightarrow$&60.55 \\
		\rowcolor[HTML]{EFEFEF}
		12&\scriptsize{PrefixedCollapsibleMapTest}&4&96\%&0&0&54&54&0.0\%&$\rightarrow$&54&0.0\%&$\rightarrow$&13.28 \\
		13&\scriptsize{TokenQueueTest}&6&69\%&18&6&152&165&8\%&{\color{ForestGreen}$\nearrow$}&152&0.0\%&$\rightarrow$&15.61 \\
		\rowcolor[HTML]{EFEFEF}
		14&\scriptsize{CharacterReaderTest}&19&79\%&71&9&309&336&8\%&{\color{ForestGreen}$\nearrow$}&309&0.0\%&$\rightarrow$&57.06 \\
		15&\scriptsize{TailDelimiterTest}&10&71\%&1&1&381&384&0.79\%&{\color{ForestGreen}$\nearrow$}&381&0.0\%&$\rightarrow$&12.90 \\
		\rowcolor[HTML]{EFEFEF}
		16&\scriptsize{LinkBufferTest}&3&48\%&12&7&66&90&36\%&{\color{ForestGreen}$\nearrow$}&66&0.0\%&$\rightarrow$&3.24 \\
		17&\scriptsize{FileNamePatternTest}&12&58\%&27&9&573&686&19\%&{\color{ForestGreen}$\nearrow$}&573&0.0\%&$\rightarrow$&25.08 \\
		\rowcolor[HTML]{EFEFEF}
		18&\scriptsize{SyslogAppenderBaseTest}&1&95\%&1&1&143&148&3\%&{\color{ForestGreen}$\nearrow$}&143&0.0\%&$\rightarrow$&7.88 \\
		19&\scriptsize{RequestBuilderAndroidTest}&2&99\%&0&0&513&513&0.0\%&$\rightarrow$&513&0.0\%&$\rightarrow$&0.04 \\
		\rowcolor[HTML]{EFEFEF}
		20&\scriptsize{CallAdapterTest}&4&94\%&0&0&55&55&0.0\%&$\rightarrow$&55&0.0\%&$\rightarrow$&7.30 \\
		\hline
		&\multicolumn{3}{l}{Low \ms}\\
		\hline
		21&\scriptsize{FieldSpecTest}&2&31\%&12&4&223&316&41\%&{\color{ForestGreen}$\nearrow$}&223&0.0\%&$\rightarrow$&4.44 \\
		\rowcolor[HTML]{EFEFEF}
		22&\scriptsize{ParameterSpecTest}&2&32\%&11&5&214&293&36\%&{\color{ForestGreen}$\nearrow$}&214&0.0\%&$\rightarrow$&3.66 \\
		23&\scriptsize{WrongNamespacesTest}&2&8\%&6&1&78&249&219\%&{\color{ForestGreen}$\nearrow$}&249&219\%&{\color{ForestGreen}$\nearrow$}&29.70 \\
		\rowcolor[HTML]{EFEFEF}
		24&\scriptsize{WrongMapperTest}&1&8\%&3&1&97&325&235\%&{\color{ForestGreen}$\nearrow$}&325&235\%&{\color{ForestGreen}$\nearrow$}&7.13 \\
		25&\scriptsize{ProgressProtocolDecoderTest}&1&16\%&2&1&18&27&50\%&{\color{ForestGreen}$\nearrow$}&23&27\%&{\color{ForestGreen}$\nearrow$}&1.30 \\
		\rowcolor[HTML]{EFEFEF}
		26&\scriptsize{IgnitionEventHandlerTest}&1&22\%&0&0&13&13&0.0\%&$\rightarrow$&13&0.0\%&$\rightarrow$&0.77 \\
		27&\scriptsize{TestICardinality}&2&7\%&0&0&19&19&0.0\%&$\rightarrow$&19&0.0\%&$\rightarrow$&2.13 \\
		\rowcolor[HTML]{EFEFEF}
		28&\scriptsize{TestMurmurHash}&2&17\%&40&2&52&275&428\%&{\color{ForestGreen}$\nearrow$}&174&234\%&{\color{ForestGreen}$\nearrow$}&2.18 \\
		29&\scriptsize{ConcurrencyTest}&2&28\%&2&0&210&342&62\%&{\color{ForestGreen}$\nearrow$}&210&0.0\%&$\rightarrow$&315.56 \\
		\rowcolor[HTML]{EFEFEF}
		30&\scriptsize{AbstractClassTest}&2&34\%&28&4&383&475&24\%&{\color{ForestGreen}$\nearrow$}&405&5\%&{\color{ForestGreen}$\nearrow$}&12.67 \\
		31&\scriptsize{AllTimeTest}&3&42\%&0&0&163&163&0.0\%&$\rightarrow$&163&0.0\%&$\rightarrow$&0.02 \\
		\rowcolor[HTML]{EFEFEF}
		32&\scriptsize{DailyTest}&3&42\%&0&0&163&163&0.0\%&$\rightarrow$&163&0.0\%&$\rightarrow$&0.02 \\
		33&\scriptsize{AttributeTest}&2&36\%&33&11&178&225&26\%&{\color{ForestGreen}$\nearrow$}&180&1\%&{\color{ForestGreen}$\nearrow$}&10.76 \\
		\rowcolor[HTML]{EFEFEF}
		34&\scriptsize{AttributesTest}&5&52\%&9&6&316&322&1\%&{\color{ForestGreen}$\nearrow$}&316&0.0\%&$\rightarrow$&6.21 \\
		35&\scriptsize{CodedDataInputTest}&1&1\%&0&0&5&5&0.0\%&$\rightarrow$&5&0.0\%&$\rightarrow$&3.58 \\
		\rowcolor[HTML]{EFEFEF}
		36&\scriptsize{CodedInputTest}&1&27\%&29&28&108&166&53\%&{\color{ForestGreen}$\nearrow$}&108&0.0\%&$\rightarrow$&0.88 \\
		37&\scriptsize{FileAppenderResilience\_AS\_ROOT\_Test}&1&4\%&0&0&4&4&0.0\%&$\rightarrow$&4&0.0\%&$\rightarrow$&0.65 \\
		\rowcolor[HTML]{EFEFEF}
		38&\scriptsize{Basic}&1&10\%&0&0&6&6&0.0\%&$\rightarrow$&6&0.0\%&$\rightarrow$&0.89 \\
		39&\scriptsize{ExecutorCallAdapterFactoryTest}&7&62\%&0&0&119&119&0.0\%&$\rightarrow$&119&0.0\%&$\rightarrow$&0.09 \\
		\rowcolor[HTML]{EFEFEF}
		40&\scriptsize{CallTest}&35&69\%&3&1&642&644&0.32\%&{\color{ForestGreen}$\nearrow$}&642&0.0\%&$\rightarrow$&52.84 \\
		\hline
	\end{tabular}
\end{table}

\subsection{Answer to \rqcandidates}
\label{subsec:test-improvement:experiment-results:rq2}

\textbf{\rqcandidates To what extent are improved test methods considered as focused?}

% presentation table
\autoref{tab:overall-results}
presents the results for RQ2, RQ3 and RQ4.% and \autoref{tab:overall-results:low_pms}
It is structured as follows.
The first column is a numeric identifier that eases reference from the text.
The second column is the name of test class to be amplified.
The third column is the number of test methods in the original test class.
The fourth column is the \ms of the original test class.
The fifth is the number of test methods generated by \dspot.
The sixth is the number of amplified test methods that met the criteria explained in \autoref{subsec:test-improvement:experiment-protocol:test-preparation}.
The seventh, eight and ninth are respectively the \ams of the original test class, the \ams of its amplified version and the absolute increase obtained with amplification, which is represented with a pictogram indicating the presence of improvement. 
The tenth and eleventh columns concern the \ams when only A-amplification is used.
The twelfth is the time consumed by \dspot to amplify the considered test class. 
The upper part of the table is dedicated to test classes that have a high \ms and the lower for the test classes that have low \ms.

For \rqcandidates{}, the considered results are in the sixth column of \autoref{tab:overall-results}.
The selection technique produces candidates that are focused in 25/26 test classes for which there are improved tests.
For instance, considering test class TypeNameTest (\#8), there are 19 improved test methods, and among them, 8 are focused per the definition and are worth considering to be integrated in the codebase.
On the contrary, for test class ConcurrencyTest (\#29), the technique cannot find any improved test method that matches the focus criteria presented in \autoref{subsubsec:test-improvement:experiment-results:rq1:selection}. 
In this case, that improved test methods kill additional mutants in 27 different locations. 
Consequently, the intent of the new amplified tests can  hardly be considered as clear.

Interestingly, for 4 test classes, even if there are more than one improved test methods, the selection technique only returns one focus candidate (\#23, \#24, \#25, \#40). 
In those cases, there are two possible different reasons:
1) there are several focused improved tests, yet they all specify the same application method (this is the case for \#40)
2) there is only one improved test method that is focused (this is the case for \#23, \#24, and \#25)

To conclude, according to this benchmark, \dspot proposes at least one and focused improved test in all but one cases. 
From the developer viewpoint, \dspot is not overwhelming it proposes a small set of suggested test changes, which are ordered, so that even with a small time budget to improve the tests, the developer is pointed to the most interesting case.

~\\
\begin{mdframed}
	\textit{\rqcandidates{}: To what extent are improved test methods considered as focused?}\\
	Answer: In 25/26 cases, the improvement is successful at producing at least one focused test method, which is important to save valuable developer time in analyzing the suggested test improvements.
\end{mdframed}
~\\

\subsection{Answer to \rqeffectiveness}
\label{subsec:test-improvement:experiment-results:rq3}

\textbf{\rqeffectiveness: To what extent do improved test classed kill more mutants than developer-written test classes?}

In 26 out of 40 cases, \dspot is able to amplify existing test cases and improves the \ms ($MS$) of the original test class.

For example, let us consider the first row, corresponding to \texttt{TypeNameTest}. 
This test class originally includes 12 test methods that kill 599 mutants. 
The improved, amplified version of this test class kills 715 mutants, \ie 116 new mutants are killed.
This corresponds to an increase of 19\% in the number of killed mutants.

First, let's discuss the amplification of the test classes that can be considered as being already good tests since they originally have a high \ms:
those good test classes are the 24 tests in \autoref{tab:overall-results}.
There is a positive increase of killed mutants for 17 cases.
This means that even when human developers write good test cases, \dspot is able to improve the quality of these test cases by increasing the number of mutants killed. 
In addition, in 15 cases, when the amplified tests kill more mutants, this goes along with an increase of the number of expressions covered with respect to the original test class.

For those 24 good test classes, the increase in killed mutants varies from 0,3\%, up to 53\%.
A remarkable aspect of these results is that \dspot is able to improve test classes that are initially extremely strong, with an original \ms of 92\% (ID:8) or even 99\% (ID:20 and ID:21). 
The improvements in these cases clearly come from the double capacity of \dspot at exploring more behaviors than the original test classes and at synthesizing new assertions.

Still looking to the upper part of \autoref{tab:overall-results} (the good test classes), focus now on the relative increase in killed mutants (column ``Increase killed''). 
The two extreme cases are \texttt{CallTest} (ID:24) with a small increase of 0.3\% and \texttt{CodeInputTest} (ID:18) with an increase of 53\%.
\texttt{CallTest} (ID:24) initially includes 35 test methods that kill 69\% of 920 covered mutants. 
Here, \dspot runs for 53 minutes and succeeds in generating only 3 new test cases that kill 2 more mutants than the original test class, and the increase in \ms is only minimal. 
The reason is that input amplification does not trigger any new behavior and assertion amplification fails to observe new parts of the program state. 
Meanwhile, \dspot succeeds in increasing the number of mutants killed by \texttt{CodeInputTest} (ID:18) by 53\%.
Considering that the original test class is very strong, with an initial \ms of 60\%, this is a very good achievement for test amplification. 
In this case, the \Iampl applied easily finds new behaviors based on the original test code.
It is also important to notice that the amplification and the improvement of the test class goes very fast here (only 52 seconds). 

One can notice 4 cases (IDs:3, 13, 15, 24) where the number of new test cases is greater than the number of newly killed mutants. 
This happens because \dspot amplifies test cases with different operators in parallel. 
While \dspot keeps only amplified test methods that kill new mutants, it happens that the same mutant is newly killed by two different amplified tests generated in parallel threads. \\
In this case, \dspot keeps both amplified test methods.

There are 7 cases with high \ms for which \dspot does not improve the number of killed mutants. 
In 5 of these cases, the original \ms is greater than 87\% (IDs: 2, 7, 12, 21, 22), and \dspot does not manage to synthesize improved inputs to cover new mutants and eventually kill them.
In some cases \dspot cannot improve the test class because they rely on an external resource (a jar file), or use utility methods that are not considered as test methods by \dspot and hence are not modified by our tool.

Now consider the tests in the lower part of \autoref{tab:overall-results}.
Those tests are weaker because they have a lower \ms. 
When amplifying weak test classes, \dspot improves the number of killed mutants in 9 out of 16 cases. 
On a per test class basis, this does not differ much from the good test classes. 
However, there is a major difference when one considers the increase itself: the increases in number of killed mutants range from 24\% to 428\%. 
Also, one can observe a very strong distinction between test classes that are greatly improved and test classes that are not improved at all (9 test classes are much improved, 7 test classes cannot be improved at all, the increase is 0\%). 
In the former case, test classes provide a good seed for amplification. 
In the latter case, there are test classes that are designed in a way that prevents amplification because they use external processes, or depend on administration permission, shell commands and external data sources; or extensively use mocks or factories; or simply very small test methods that do not provide a good potential to \dspot to perform effective amplification.

~\\
\begin{mdframed}
	\textit{\rqeffectiveness: To what extent do improved  test classes kill more mutants than manual test classes?}\\
	Answer: In this quantitative experiment on automatic test improvement, \dspot significantly improves the capacity of test classes at killing mutants in 26 out 40 of test classes, even in cases where the original test class is already very strong. 
	Automatic test improvement works particularly well for weakly tested classes (lower part of \autoref{tab:overall-results}): the \ms of three classes is increased by more than 200\%.
\end{mdframed}
~\\
The most notable point of this experiment is that there are considered tests that are already really strong (\autoref{tab:overall-results}), with \ms in average of 78\%, with the surprising case of a test class with 99\% \ms that \dspot is able to improve. 

\subsection{Answer to \rqAmplVersusIAmpl}
\label{subsec:test-improvement:experiment-results:rq4}

\textbf{What is the contribution of \Iampl and \Aampl to the effectiveness of automated test improvement?}

The relevant results are reported in the tenth and eleventh column of \autoref{tab:overall-results}.
They give the \ams and the relative increase of the number of killed mutants when only using \Aampl.

For instance, for \texttt{TypeNameTest} (first row, id \#1), using only \Aampl kills 599 mutants, which is exactly the same number of the original test class. 
In this case, both the absolute and relative increase are obviously zero.
On the contrary, for \texttt{WrongNamespacesTest} (id \#27), using only \Aampl is very effective, it enables \dspot to kill 249 mutants, which, compared to the 78 originally killed mutants, represents an improvement of 219\%. 

Now, when aggregating over all test classes, the results indicate that \Aampl only is able to increase the number of mutants killed in 7 / 40 test classes. 
Increments range from 0.31\% to 13\%. 
Recall that when \dspot runs both \Iampl{} and \Aampl, it increases the number of mutants killed in 26 / 40 test classes, which shows that it is indeed the combination of \Aampl and \Iampl which is effective.

Note that \Aampl performs as well as \Iampl + \Aampl in only 2/40 cases (ID:27 and ID:28). 
In this case, all the improvement comes from the addition of new assertions, and this improvement is dramatic (relative increase of 219\% and 235\%).

The limited impact of \Aampl alone has several causes. 
First, many assertions in the original test cases are already good and precisely specify the expected behavior for the test case.
Second, it might be due to the limited observability of the program under test (\ie, there is a limited number of points where assertions over the program state can be expressed).
Third, it happens when one test case covers global properties across many methods: test \#28 \texttt{WrongMapperTest} specifies global properties, but is not well suited to observe fine grained behavior with additional assertions. 
This latter case is common among the weak test classes of the lower part of \autoref{tab:overall-results}.

~\\

\begin{mdframed}
	\textit{\rqAmplVersusIAmpl: What is the contribution of \Iampl{} and \\\Aampl{} to the effectiveness of test amplification?}\\
	Answer: The conjunct run of \Iampl{} and \Aampl{} is the best strategy for \dspot{} to improve manually-written test classes. 
	This experiment has shown that \Aampl{} is ineffective, in particular on tests that are already strong.
\end{mdframed}
~\\

To the best of my knowledge, this experiment is the first to evaluate the relative contribution of \Iampl and \Aampl to the effectiveness of automatic test improvement.

\section{Threats to Validity}
\label{sec:test-improvement:threats}

\textbf{\rqpullrequest{}}
The major threat to \rqpullrequest is that there is a potential bias in the acceptance of the proposed pull requests.
For instance, if I propose pull requests to colleagues, they are more likely to merge them.
However, this is not the case here.
In this evaluation, I am unknown to all considered projects. 
The developers who study the \dspot pull requests are independent from our group and social network.
Since I was unknown for the pull request reviewer, this is not a specific bias towards acceptance or rejection of the pull request.

\textbf{\rqcandidates{}}
The technique used to select focused candidates is based on the proportion of mutant killed and the absolute number of modification done by the amplification. 
However, it may happen that some improvements that are not focused per our definition would still be considered as valuable by developers. 
Having such false negative is a potential threat to validity.

\textbf{\rqeffectiveness{}}
A threat to \rqeffectiveness{} relates to external validity: if the considered projects and tests are written by amateurs, the findings would not hold for serious software projects.
However, the experimentation only considers real-world applications, maintained by professional and esteemed open-source developers.
This means that considered tests are arguably among the best of the open-source world, aiming at as strong construct validity as possible.

\textbf{\rqAmplVersusIAmpl{}.}
The main threat to \rqAmplVersusIAmpl{} relates to internal validity: since the results are of computational nature, a bug in the implementation or experimental scripts may threaten the findings. 
All the code is publicly-available for other researchers to reproduce the experiment and spot the bugs, if any.

\textbf{Oracle.}
\dspot generates new assertions based on the current behavior of the program. 
If the program contains a bug, the resulting amplified test methods would enforce this bug. 
This is an inherent threat, inherited from \cite{Xie2006}, which is unavoidable when no additional oracle is available, but only the current version of the program. 
To that extent, the best usage of \dspot is to improve the test suite of a supposedly almost correct version of the program.

\section{Conclusion}
\label{sec:test-improvement:conclusion}

This first evaluation of \dspot gave 2 main results:

1) 19 amplified test methods have been proposed to be integrated in test suites from open-source projects. 
The developers of these projects reviewed amplified test methods, proposed in the form of pull requests.
14 of them have been merged permanently in the test suite of projects.
It means that developers value amplified test methods produced by \dspot.
It also means that amplified test methods, obtained using \dspot, are increasing the developers' confidence in the correctness of their program;

2) 40 test classes have been amplified to improve their \ms.
26 of them result with an actual improvement of the \ms.
This shows that \dspot is able to improve existing test suites.

In this chapter, the \ms has been used to amplified test methods.
The \ms is a measure of the test suites quality to detect small behavioral changes, as mutants emulate them.
However, the behavioral changes introduces by mutants may be different from those in real commits.
That is to say, it does not show any evidence that \dspot would be able to detect behavioral change introduced by commits, which are typically larger, more complex and significant.

In the next chapter, I investigate the capacity of \dspot to improve existing test methods in order to detect real behavioral changes, introduced by commits.
To do so, I confront \dspot to real modifications done by developers on their code base from \gh.
In addition to this, the next chapter exposes an enhancement of \dspot's usage and puts it in the continuous integration.