\usepackage{amsmath,amssymb}
\usepackage[latin1]{inputenc}
\usepackage[T1]{fontenc}
\usepackage[left=1.5in,right=1.3in,top=1.1in,bottom=1.1in,includefoot,includehead,headheight=13.6pt]{geometry}
\renewcommand{\baselinestretch}{1.04}
\usepackage{lipsum}
\usepackage[nottoc, notlof, notlot]{tocbibind}
\usepackage{minitoc}
\setcounter{minitocdepth}{2}
\usepackage{aecompl}
\usepackage{listings}
\usepackage{glossaries}

\lstdefinestyle{java}{
  language=Java,
  showspaces=false,
  showtabs=false,
  breaklines=true,
  showstringspaces=false,
  breakatwhitespace=true,
  commentstyle=\color{green},
  keywordstyle=\color{blue},
  stringstyle=\color{red},
  basicstyle=\scriptsize,
  moredelim=[il][\textcolor{grey}]{$$},
  moredelim=[is][\textcolor{grey}]{\%\%}{\%\%}
}

\usepackage{color}
\usepackage{verbatim}
\usepackage{xcolor}
\definecolor{GreyRowTable}{HTML}{EEEEEE}
\usepackage{mdframed} 
\usepackage[intoc]{nomencl}
\renewcommand{\nomname}{List of Abbreviations}
\makenomenclature
\usepackage{ifpdf}
\usepackage{url}
\ifpdf
  \usepackage[pdftex]{graphicx}
  \DeclareGraphicsExtensions{.jpg}
  \usepackage[a4paper,pagebackref,hyperindex=true]{hyperref}
\else
  \usepackage{graphicx}
  \DeclareGraphicsExtensions{.ps,.eps}
  \usepackage[a4paper,dvipdfm,pagebackref,hyperindex=true]{hyperref}
\fi
\usepackage[nolist,nohyperlinks]{acronym}
\renewcommand*{\backref}[1]{}
\renewcommand*{\backrefalt}[4]{%
\ifcase #1 %
(Not cited.)%
\or
(Cited on page~#2.)%
\else
(Cited on pages~#2.)%
\fi}
\renewcommand*{\backrefsep}{, }
\renewcommand*{\backreftwosep}{ and~}
\renewcommand*{\backreflastsep}{ and~}
\usepackage{color}
\definecolor{linkcol}{rgb}{0,0,0.4} 
\definecolor{citecol}{rgb}{0.5,0,0} 
\hypersetup
{
bookmarksopen=true,
pdftitle="thesis",
pdfauthor="Benjamin Danglot", 
pdfsubject="Behavioral Changes Detection", %subject of the document
%pdftoolbar=false, % toolbar hidden
pdfmenubar=true, %menubar shown
pdfhighlight=/O, %effect of clicking on a link
colorlinks=true, %couleurs sur les liens hypertextes
pdfpagemode=None, %aucun mode de page
pdfpagelayout=SinglePage, %ouverture en simple page
pdffitwindow=true, %pages ouvertes entierement dans toute la fenetre
linkcolor=linkcol, %couleur des liens hypertextes internes
citecolor=citecol, %couleur des liens pour les citations
urlcolor=linkcol %couleur des liens pour les url
}
\setcounter{secnumdepth}{3}
\setcounter{tocdepth}{2}
\newcommand{\pd}[2]{\frac{\partial #1}{\partial #2}}
\def\abs{\operatorname{abs}}
\def\argmax{\operatornamewithlimits{arg\,max}}
\def\argmin{\operatornamewithlimits{arg\,min}}
\def\diag{\operatorname{Diag}}
\newcommand{\eqRef}[1]{(\ref{#1})}

\usepackage{rotating}
\usepackage{fancyhdr}
\pagestyle{fancy}
\fancyfoot{}
\fancyhead[LE,RO]{\bfseries\thepage} 
\fancyhead[RE]{\bfseries\nouppercase{\leftmark}}  
\fancyhead[LO]{\bfseries\nouppercase{\rightmark}} 
\let\headruleORIG\headrule
\renewcommand{\headrule}{\color{black} \headruleORIG}
\renewcommand{\headrulewidth}{1.0pt}
\usepackage{colortbl}
\arrayrulecolor{black}
\fancypagestyle{plain}{
  \fancyhead{}
  \fancyfoot{}
  \renewcommand{\headrulewidth}{0pt}
}
\usepackage{algorithm}
\usepackage{algorithmicx}
\usepackage{algpseudocode}
\makeatletter
\def\cleardoublepage{\clearpage\if@twoside \ifodd\c@page\else%
  \hbox{}%
  \thispagestyle{empty}%              % Empty header styles
  \newpage%
  \if@twocolumn\hbox{}\newpage\fi\fi\fi}

\makeatother
\newcommand{\reviewtimetoday}[2]{\special{!userdict begin
    /bop-hook{gsave 20 710 translate 45 rotate 0.8 setgray
      /Times-Roman findfont 12 scalefont setfont 0 0   moveto (#1) show
      0 -12 moveto (#2) show grestore}def end}}
\newenvironment{maxime}[1]
{
\vspace*{0cm}
\hfill
\begin{minipage}{0.5\textwidth}%
%\rule[0.5ex]{\textwidth}{0.1mm}\\%
\hrulefill $\:$ {\bf #1}\\
%\vspace*{-0.25cm}
\it 
}%
{%

\hrulefill
\vspace*{0.5cm}%
\end{minipage}
}

\let\minitocORIG\minitoc
\renewcommand{\minitoc}{\minitocORIG \vspace{1.5em}}

\usepackage{multirow}

\newenvironment{bulletList}%
{ \begin{list}%
	{$\bullet$}%
	{\setlength{\labelwidth}{25pt}%
	 \setlength{\leftmargin}{30pt}%
	 \setlength{\itemsep}{\parsep}}}%
{ \end{list} }

\newtheorem{definition}{Definition}
\newtheorem{fact}{Fact}
\renewcommand{\epsilon}{\varepsilon}

% centered page environment

\newenvironment{vcenterpage}
{\newpage\vspace*{\fill}\thispagestyle{empty}\renewcommand{\headrulewidth}{0pt}}
{\vspace*{\fill}}

\usepackage{xspace}
\usepackage{lettrine}

% Landscape package
\usepackage{pdflscape}
% use \begin{landscape} and \end{landscape}

\newdimen\origiwspc%
\newdimen\origiwstr%
\origiwspc=\fontdimen2\font% original inter word space
\origiwstr=\fontdimen3\font% original inter word stretch

%
%   Can play with the following value to change words spacing
%
\fontdimen2\font=0.2ex% inter word space
\fontdimen3\font=0.1em% inter word stretch
\fontdimen2\font=\origiwspc% (original) inter word space
\fontdimen3\font=\origiwstr% (original) inter word stretch

% line spacing
\usepackage{setspace}
\renewcommand{\baselinestretch}{1.2}
%{\setstretch{1.0
%text
%}

\usepackage{afterpage}

\newcommand{\TODO}[1]{\textcolor{red}{#1}\GenericWarning{}{LaTeX Warning: TODO: #1}}\newcommand\todo\TODO

% ==========================================================
%		DSPOT 		COMMANDS
% ==========================================================

\newcommand{\algorithmautorefname}{Algorithm}
\newcommand{\Iampl}{\emph{I-Amplification}\xspace}
\newcommand{\Aampl}{\emph{A-Amplification}\xspace}
\newcommand{\dspot}{DSpot\xspace}
\newcommand{\etal}{\textit{et al.}\xspace}
\newcommand{\ie}{\textit{i.e.}\xspace}
\newcommand{\eg}{\textit{e.g.}\xspace}
\newcommand{\gh}{GitHub\xspace}
\newcommand{\pitest}{Pitest\xspace}
\newcommand{\xwiki}{XWiki\xspace}
\newcommand{\ms}{mutation score\xspace}
\newcommand{\ams}{number of killed mutants\xspace}
\newcommand{\junit}{JUnit\xspace}

\newcommand*\rotvertical{\rotatebox{-90}}
\newcommand*\rotverticalinv{\rotatebox{90}}
\newcommand*\rotlarge{\rotatebox{70}}
\newcommand*\rothalf{\rotatebox{45}}
\newcommand{\rot}[1]{\rotatebox{#1}}

\definecolor{ForestGreen}{RGB}{34,139,34}

\usepackage{etex}
\usepackage{graphicx}
\usepackage{xcolor}
\usepackage{listings}  
\usepackage{amsmath}
\usepackage{balance}
\usepackage{tabularx}
\usepackage{booktabs}% for \midrule
\usepackage{mdframed} 
\usepackage{verbatim} % adds environment for commenting out blocks of text & for better verbatim
%\usepackage{enumitem}
\usepackage{paralist}
\usepackage{listings}
\usepackage{courier}
\usepackage{xspace}
\usepackage{balance}
\usepackage{multirow}
\definecolor{ForestGreen}{RGB}{34,139,34}
\definecolor{lightgray}{RGB}{225,225,225}
\usepackage[]{pdfcomment}
\usepackage{float}
\usepackage{lscape}
\usepackage{rotating}
\usepackage{adjustbox}

% DCI PACKAGES

\usepackage{etex}
\usepackage{graphicx}
\usepackage{xcolor}
%[colorlinks,linkcolor=blue,citecolor=blue,urlcolor=blue]
\usepackage{listings}  
\usepackage{amsmath}
\usepackage{balance}
\usepackage{tabularx}
\usepackage{booktabs}% for \midrule
\usepackage{mdframed} 
\usepackage{verbatim} % adds environment for commenting out blocks of text & for better verbatim
\usepackage{paralist}
\usepackage{listings}
\usepackage{courier}
\usepackage{xspace}
\usepackage{balance}
\usepackage{multirow}
\definecolor{ForestGreen}{RGB}{34,139,34}
\usepackage[]{pdfcomment}
\usepackage{lscape}

%\usepackage{hyperref}
\usepackage[inline]{enumitem}
\usepackage{amsmath}

\usepackage{amssymb}% http://ctan.org/pkg/amssymb
\usepackage{pifont}% http://ctan.org/pkg/pifont
\newcommand{\cmark}{{\color{ForestGreen}\ding{51}}\xspace}%
\newcommand{\lcmark}{{\textbf{\color{green}\ding{72}}}\xspace}%
\newcommand{\xmark}{{\color{red}\ding{55}}\xspace}%

\newcommand{\java}{Java\xspace}
\newcommand{\ExAmplifier}{\textbf{Ex2Amplifier}\xspace}
\newcommand{\head}{\emph{Pull request}\xspace}
\newcommand{\base}{\emph{base}\xspace}
\newcommand{\aampl}{\textsc{AAMPL}\xspace}
%\newcommand{\seci}{\textbf{SECI}\xspace}
\newcommand{\catg}{\textbf{CATG}\xspace}
\newcommand{\sbampl}{\textsc{SBAMPL}\xspace}
\newcommand{\nbProjects}{4\xspace}
\newcommand{\nbPullRequests}{36\xspace}

\lstset{ %
	%  backgroundcolor=\color{white},   % choose the background color; you must add \usepackage{color} or \usepackage{xcolor}
	basicstyle=\small\ttfamily,        % the size of the fonts that are used for the code
	captionpos=b,                    % sets the caption-position to bottom
	commentstyle=\small\color{ForestGreen},    % comment style
	escapeinside={(*@}{@*)},          % if you want to add LaTeX within your code
	keywordstyle=\color{blue},       % keyword style
	stringstyle=\color{purple},       % keyword style
	numberstyle=\tiny\color{black}, % the style that is used for the line-numbers
	stepnumber=1,                    % the step between two line-numbers. If it's 1, each line will be numbered
	title=\lstname,                   % show the filename of files included with \lstinputlisting; also try caption instead of title
	language=Java,
	breakatwhitespace=false,         % sets if automatic breaks should only happen at whitespace
	breaklines=true,                 % sets automatic line breaking
	extendedchars=true,              % lets you use non-ASCII characters; for 8-bits encodings only, does not work with UTF-8
	%frame=single,                    % adds a frame around the code
	showspaces=false,                % show spaces everywhere adding particular underscores; it overrides 'showstringspaces'
	showstringspaces=false,          % underline spaces within strings only
	showtabs=false,                  % show tabs within strings adding particular underscores
	tabsize=1,                       % sets default tabsize to 2 spaces
	framexleftmargin=15pt,
	frame = single
}

\newcommand{\rev}[1]{\textcolor{orange}{#1}\GenericWarning{}{LaTeX Warning: Revision: #1}}

\newcommand{\RQ}[2]{RQ{#1}: #2}

%%% Hilighting commands
\definecolor{red}{rgb}{1,0,0}
\definecolor{green}{rgb}{0,1,0}
\definecolor{blue}{rgb}{0,0,1}
\definecolor{cyan}{rgb}{0.4,1,1}
\definecolor{orange}{rgb}{0.9,0.5,0}
\definecolor{dkgreen}{rgb}{0,0.6,0}
\definecolor{gray}{rgb}{0.5,0.5,0.5}
\definecolor{purple}{rgb}{0.58,0,0.82}
\usepackage{times}

%Hilighting commands:
\newcommand{\todox}[1]{{\color{red}\bf\em TODO: #1}}
\newcommand{\commentx}[1]{{\color{purple}\bf\em Comment: #1}}
\newcommand{\done}[1]{{\color{blue}\bf\em DONE: #1}}\newcommand\DONE\done
\newcommand{\hll}[1]{\textcolor{orange}{#1}}
\newcommand{\hlll}[1]{\textcolor{dkgreen}{#1}}

\newcommand{\DCIA}{DCI$_{AAMPL}$\xspace}
\newcommand{\DCII}{DCI$_{SBAMPL}$\xspace}

% enable line breaking in texttt 
\usepackage[htt]{hyphenat}

\newenvironment{chaptersummary}
{
	\begin{center}
		\begin{mdframed}
		\begin{large}
		}
		{ 
		\end{large}
		\end{mdframed}
	\end{center}
}
%--------------------------------------------------
